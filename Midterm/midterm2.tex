\documentclass[12pt]{article}
\usepackage{enumerate}

% Increase margins (manually) 
% Sides (odd- and even-numbered pages)
   \addtolength{\oddsidemargin}{-0.875in}
   \addtolength{\evensidemargin}{-0.875in}
   \addtolength{\textwidth}{1.75in}
% Top/bottom
   \addtolength{\topmargin}{-0.875in}
   \addtolength{\textheight}{1.75in}

\begin{document}

\textbf{ASTR 621: Planetary Science}

\begin{enumerate}%[(1)]
\item Jean's Mass Equation 
  \begin{equation}
    M_J ~ \frac{kT}{Gm\mu}^{3/2}\frac{1}{sqrt(\rho)}
  \end{equation}

\item Observational Characteristics across a wide wavelength 
(UV to IR) range
\item Free fall time
\begin{itemize}
  \item something  
\end{itemize}
\item Toomre Stability Critera/Parameter
\item Viscous processes
\begin{itemize}
  \item $\nu$ = kinematic viscosity  
  \item viscous couple
  \item viscosity transports angular momentum outward
  \item time scale for viscous mixing much larger than that of star
  formation
\end{itemize}

\item Scale Height equation
  \begin{equation}
    \rho(z) = \rho(z=0)e^{(-z^2/H^2_z)}
  \end{equation}
  for an isothermal disk (constant temperature). \\
  How do observations of disks around stars in our galaxy support this
  derived scale geight inference, and what do these observations
  indicate about the disk's mass relative to the central protostar's
  mass?
\item Discuss the importance of dust grains within the nebula's
\begin{itemize}
  \item structure (radially and vertical `z' direction) \\
  \item viscosity (mixing), both radially and in the `z' direction \\
  Dust grains are the most probable source of opacity within cold
  regions of the nebula. Nebula loses thermal energy in the z
  direction. 

  \item `z' direction temperature (disk plane temperature vs `surface'
  temperature \\
  \item mass structures \\
\end{itemize}
\item Describe the `minimum mass solar nebula' concept including what
it implies rgarding material type and quantity available when the
solar nebula formed and how its value is arrived at;
  \begin{itemize}
    \item small enough mass so that its gravitational effect <<
    central core's (justified by observed 'bowtie' disk).
    \item Must account for all of the non-solar mass in the solar
    system (assuming formation process was 100\% efficient in
    retaining the available rock (~0.5\%) and ice (~1\%).\\
    Maybe magnetic field responsible for `missing' angular momentum?
  \end{itemize}
\item Gravitational instabilities
\item Physical/environmental conditions involved in grain formation
\begin{itemize}
  \item Gibbs free energy
  \item Dust in the nebula
  \item Dust grain growth requires:
    \begin{itemize}
      \item non-trivial supersaturation (cooling in excess of what
      might be expected)
      \item heterogeneous nucleation (condensation around a
      pre-existing particle...such as?)
    \end{itemize}
\end{itemize}
\item grain size-dependent motions within the nebular disk
\item planetesimal and protoplanet growth in the disk
\item Physical characteristics that characterize 'core accretion' vs.
`gas instability' growth of Jovian-type planets
\item core accretion
\item Exoplanet detection
  \begin{itemize}
    \item transit - change in light curve
    \item radial velocity - stellar reflex motion
    \item direct imaging - want planet close to star
    \item microlensing - opportunistic, confirm other ways
    \item astrometry - variation of star's position
    \item timing - first planet found this way!
    \item radio emission - (proposed)
    \item aliens
  \end{itemize}
\item planet 'types'
\item Energetic processes envolved with planet growth via material
accretion

\end{enumerate}

\end{document}
