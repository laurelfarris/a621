\documentclass[12pt]{article}
\usepackage[margin=1in]{geometry}
\usepackage{enumerate}
\usepackage{enumitem}


\setitemize{noitemsep}
%\setlist{itemize}{noitemsep}

\begin{document}

\textbf{ASTR 621: Planetary Science}

\begin{enumerate}
    \item Stevenson's 'laws' of origins studies
    \item ISM properties
        \begin{enumerate}
            \item Important observations:
                \begin{enumerate}
                    \item Only molecules with a permanent dipole moment can be detected by
                        their rotational emission lines, typically at mm and sub-mm
                        wavelengths.
                    \item H$_{2}$, C$_{2}$ seen by their electronic absorption lines at
                        optical wavelengths against bright background stars.
                    \item Cannot observe CH$_{4}$, N$_{2}$, C$_{2}$H$_{2}$ since no dipole
                    \item Earth's atmosphere prevents H$_{2}$O, O$_{2}$, CO$_{2}$
                \end{enumerate}
            \item \textbf{ISM /= protosolar nebula!}
            \item Magnetic field
            \item Multiple star systems -8-
            \item Angular momenta of stars -9-
            \item Missing mass problem -9-
            \item Young Stellar Objects (YSOs) -13-
                Best studied in the IR because they are surrounded by dust,
                the objects themselves are most luminous in the IR, and there
                is an increased contrast against the nebular background.
                T-Tauri stars -15-
                \begin{itemize}
                    \item Found in molecular clouds and OB associations.
                    \item $M \sim M_{\odot}$
                    \item irregular variability
                    \item UV excess
                    \item IR excess
                    \item High mass outflow
                \end{itemize}
                These features are likely explained by \textbf{circumstellar disk}
        \end{enumerate}
        As a result, our information of what may be the dominant C, O, N
        molecules is indirect and incomplete.
    \item Gravitational Collapse -18-
        \begin{enumerate}
            \item Jean's Criterion: Uniform, homogeneous gas at rest,
                infinite extent, assume adiabatic
                \begin{itemize}
                    \item $w^2 = c^2k^2 - 4\pi G\rho_o$;
                        $w$ = frequency\ldots the higher this is, the more stable the
                        cloud is against collapse.
                    \item $n_{crit} = 10^{6}(T/20)^{3}(M_{\odot}/M)^{2}\;[\mathrm{cm}^{-3}]$
                \end{itemize}
            \item Comparison between observation and theory:
                Rotation and magnetic fields must preserve clouds since we see
                too many for the short timescales predicted by theory to be
                true.
            \item Fragmentation
            \item Collapse modification
                \begin{itemize}
                    \item $w^2 = c^2k^2 + 2({\omega}k)^2/k^2  - 4\pi G\rho_o$
                    \\
                    $w^2$ is increased with a third term (always positive or
                    zero) that accounts for
                    rotation, which prevents collapse.
                \end{itemize}
        \end{enumerate}
    \item Hayashi model -33-
        \begin{itemize}
            \item adiatic contraction
            \item hydrostatic equilibrium
        \end{itemize}
    \item Accretion disk resulting from nebular collapse -65-
        \begin{itemize}
            \item Viscosity transports angular momentum outward
            \item Mass near core is transported inward;
                mass farther is transported outward
        \end{itemize}
    \item Opacity of the primordial solar nebula -71-
        \begin{itemize}
            \item Effect of opaticty on vertical (z) direction
        \end{itemize}
    \item Radial temperature gradient -75-
    \item Dust in the Nebula -113-
    \item Particle Motion in the nebula -124-
    \item Processes that affect particle radial location in the nebula -129-
        \begin{enumerate}
            \item gas drag
            \item radiation force
            \item Poynting-Robertson Drag
        \end{enumerate}
    Yarkovsky effect: experienced by objects large enough to
    experience spatial variations in surface temperature \\
    $-->$ Spiraling radially outward!

    \item{The Hill radius} \\
    Range of gravitational influence of planet orbiting a star.
    \item{Planetary Migration} \\
    \begin{itemize}
      \item Type I: linear, low-mass planets (< M$_{Saturn}$)
      \item Type II: non-linear, gaps, larger masses
    \end{itemize}

    \item{Comets}


\end{enumerate}

\end{document}









